%\title{Resultados}
\documentclass{article}
\usepackage[spanish, es-tabla]{babel}
\usepackage[utf8]{inputenc}

%\usepackage[utf8x]{inputenc}
\usepackage[T1]{fontenc}


%% Sets page size and margins
\usepackage[letterpaper,top=3.5cm,bottom=2cm,left=3cm,right=3cm,marginparwidth=1.75cm, headsep = 70pt]{geometry}

%% Useful packages
\usepackage{amsmath}
\usepackage{float}
\usepackage{graphicx}
\usepackage{multirow}
\usepackage{booktabs, makecell}
\usepackage[table,xcdraw]{xcolor}
\usepackage[colorinlistoftodos]{todonotes}
\usepackage[colorlinks=true, allcolors=blue]{hyperref}
\usepackage[hang, small,up,textfont=it,up]{caption} 
\usepackage{fancyhdr}
\usepackage{subcaption}


%cabecera y pies
\pagestyle{fancy}
\fancyhf{}
\rhead{\includegraphics[width=0.12\textwidth]{SAMU_e.png}}
\lhead{
Dr E. Céspedes G., Tecnologías SAMU \\
SAMU V Región \\
Octubre 2019}
\rfoot{P\'agina \thepage}

% titulo documento
\title{Perfil de la atención de urgencia en contexto de manifestaciones ciudadanas masivas }
\author{Dr E Céspedes-González}
\date{Octubre 2017}


\begin{document}
\maketitle

\section{Introducción}

La medicina de urgencia concentra entre otras ramas la medicina de desastres o los comportamientos poblacionales en relación a la salud de grandes movimientos de personas, migraciones, situaciones de guerra, etc....


Constantemente en relación a la contingencia se han visto eventos suceptibles de ser estudiados y caracterizados. Lo anterior sumado a la disponibilidad de bases de datos generadas por elementos de trabajo cotidianos en la atención de urgencias.

El objetivo de este trabajo es describir el comportamiento de las atenciones de urgencias en un servicio de salud en contexto de estado nacional en contingencia dado movimientos sociales masivos.


\section{Metodología}
Estudio de corte tranversal retrospectivo. Se descargó del sistema informático utilizado para la atención de urgencia del SSVQ planillas que permitieran conocer características epidemiológicas de los pacientes y  características de las atenciones de urgencia (Triage, destino del paciente, como llegaron a la UE, etc....). Se consolidó información en una única base de datos en Python3 que luego se analizó con ese mismo lenguaje con distintas librerías para el rocesamiento y estudio de datos, finalmente se realizó un análsis estadístico con el mismo lenguaje.

Se incluyeron todas las atenciones de urgencia de 11 hospitales del Servicio de Salud, lo que compromete 18 Comunas, 7.506Km cuadrados y 1050338 personas (śin difernciar por previsión) según el censo del 2017. Cabe destacar que las atenciones en servicios de urgencia municipales o privados no están incluídos en el estdio.

La fecha de análisis de interés de las manifestaciones masivas fue del viernes 18 de Octubre 00.00h al XXXXXX. Otros datos para comparar se obtuvieron de una base de datos similar que comprende desde Enero hasta Septiembre del año 2019.

\section{Resultados}

Se estudió un total de X días....., Se registraron en total 12510 atenciones de urgencia en el periodo estudiado, que al axtrapolarlo a 30 días corresponde a 41700 atenciones. En la Tabla 1 se muestra el detalle por hospital y la normalización a los 30 días, también se muestran datos de un periodo regular para poder comparar.

En la Tabla 2 se muestra la frecuencia relativa de categorización para cada periodo estudiado. Se preenta también la diferencia absoluta entre los porcentajes que representa cada categoría de la categorización ESI.


\begin{table}[H]
\centering
\begin{tabular}{@{}lcccc@{}}
\toprule
\textbf{Hospital}         & \textbf{Usual} & \textbf{Periodo estudio} & \textbf{Relación} \\ \midrule
Hospital de Viña          & 14735,20       & 10470                    & 71,05\%           \\
Hospital de Quillota      & 7803,10        & 6410                     & 82,15\%           \\
Hospital de Quilpué       & 7130,70        & 5856,7                   & 82,13\%           \\
Hospital de Quintero      & 4666,60        & 3563,3                   & 76,36\%           \\
Hospital de Calera        & 4062,40        & 3386,7                   & 83,37\%           \\
Hospital de Limache       & 3946,60        & 3186,7                   & 80,75\%           \\
Hospital de La Ligua      & 3466,60        & 3063,3                   & 88,37\%           \\
Hospital de Villa Alemana & 3182,10        & 2946,7                   & 92,60\%           \\
Hospital de Cabildo       & 2033,80        & 1836,7                   & 90,31\%           \\
Hospital de Petorca       & 1090,30        & 980                      & 89,88\%           \\ 
Total                     & 52.117,40      & 41.700,10                & 80,01\%           \\ \bottomrule
\end{tabular}
\caption{Cantidad de atenciones por hospital al normalizar las atenciones a 30 días}
\end{table}

\begin{figure}[h]
\centering
\includegraphics[width=0.8\linewidth,]{../producto/atenciones_porhospital.png}
\caption{Cantidad de atenciones que en los distintos hospitales del Servicio de Salud Viña del Mar Quillota en el periodo de manifestaciones ciudadanas}
\label{fig:image2}
\end{figure}

\begin{table}[H]
\centering
\begin{tabular}{@{}lccc@{}}
\toprule
\textbf{Triage} & \textbf{Usual} & \textbf{Periodo estudio} & \textbf{Incremento porcentual} \\ \hline
Triage 1        & 0,39\%         & 0,44\%                   & 12,40\%                        \\
Triage 2        & 18,80\%        & 19,96\%                  & 6,13\%                         \\
Triage 3        & 28,46\%        & 28,17\%                  & -1,02\%                        \\
Triage 4        & 50,27\%        & 48,38\%                  & -3,76\%                        \\
Triage 5        & 2,08\%         & 3,06\%                   & 47,20\%                        \\ \hline
\end{tabular}
\caption{Cantidad de atenciones y frecuancia relativa por Triage ESI. Sólo se presentan atenciones de los 3 hospitales que tienen instaurado el Triage ESI (Hospital de Viña, de Quillota y Quilpué)}
\end{table}

\begin{figure}[h]
\centering
\includegraphics[width=0.8\linewidth,]{../producto/atenciones_133.png}o
\caption{Cantidad de atenciones que involucró motivo de consulta 'Constatación de lesiones' o que Carabineros de Chile o Policía de Investigaciones intervinieron en la admisión o custodiaron a paciente luego del alta}
\label{fig:image2}
\end{figure}

Se analizó la base de datos y se seleccionó un subgrupo de pacientes de interés: aquellos que dentro del motivo de consulta estaba registrado 'Constatación de lesiones' o fueron ingresados  a la urgencia por personal policial o el alta hospitalaria involucró custodia policial. El grupo de interés contó con 802 consultas en las unidades de Urgencias.

Los paciente que corresponían a este grupo eran principalmente hombres jóvenes con lesiones de carácterter leve. En la Tabla X se presenta un resumen de las diferencias epidemiológicas y del resultado de las atenciones del grupo de interés en un periodo usual y el el estudiado.


\begin{table}[H]
\begin{tabular}{@{}llccc@{}}
\toprule
                          &                               & \textbf{Usual} & \textbf{Periodo estudio} & \textbf{Incremento porcentual} \\ \midrule
                          & Hombres, \%                   & 67,00\%        & 72,00\%                  & 7,46\%                         \\
                          & Edad, mediana (DS)            & 34,4(15,8)     & 28,9 (13,2)              &                                \\ \cmidrule(l){1-5}
\multirow{4}{*}{Lesiones} & Sin lesiones, \%              & 46,50\%        & 38,90\%                  & -16,34\%                       \\ 
                          & Lesiones Leves, \%            & 47,70\%        & 53,40\%                  & 11,95\%                        \\
                          & Lesiones mediana gravedad, \% & 3,30\%         & 4,60\%                   & 39,39\%                        \\
                          & Lesiones Graves, \%           & 2,30\%         & 2,91\%                   & 26,52\%                        \\ \cmidrule(l){1-5}
\multirow{3}{*}{Destino}  & Hospitalizado,  \%            & 0,80\%         & 0,25\%                   & -68,75\%                       \\ 
                          & Alta, \%                      & 31,09\%        & 24,08\%                  & -22,55\%                       \\
                          & Custodia policial, \%         & 64,10\%        & 72,30\%                  & 12,79\%                        \\ \cmidrule(l){1-5} 
\end{tabular}
\caption{Diferencias epidemiológicas y de la finalización de la atencion en pacientes de interés (que consultan por 'Constatación de lesiones' o fueron ingresados a la unidad de emergencia por personal policial o fueron egresados con custodia policial)}
\end{table}




\end{document}

