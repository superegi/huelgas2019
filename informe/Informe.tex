%\title{Resultados}
\documentclass{article}
\usepackage[spanish, es-tabla]{babel}
\usepackage[utf8]{inputenc}

%\usepackage[utf8x]{inputenc}
\usepackage[T1]{fontenc}


%% Sets page size and margins
\usepackage[letterpaper,top=3.5cm,bottom=2cm,left=3cm,right=3cm,marginparwidth=1.75cm, headsep = 70pt]{geometry}

%% Useful packages
\usepackage{amsmath}
\usepackage{float}
\usepackage{graphicx}
\usepackage{multirow}
\usepackage{booktabs, makecell}
\usepackage[table,xcdraw]{xcolor}
\usepackage[colorinlistoftodos]{todonotes}
\usepackage[colorlinks=true, allcolors=blue]{hyperref}
\usepackage[hang, small,up,textfont=it,up]{caption} 
\usepackage{fancyhdr}


%cabecera y pies
\pagestyle{fancy}
\fancyhf{}
\rhead{\includegraphics[width=0.12\textwidth]{SAMU_e.png}}
\lhead{
Dr E. Céspedes G., Tecnologías SAMU \\
SAMU V Región \\
Octubre 2019}
\rfoot{P\'agina \thepage}

% titulo documento
\title{Perfil de la atención de urgencia en contexto de manifestaciones ciudadanas }
\author{Dr E Céspedes-González}
\date{Septiembre 2017}


\begin{document}
\maketitle

\section{Introducción}

La medicina de urgencia concentra entre otras ramas de estudio la 

Constantemente en relación a la contingencia se han visto eventos suceptibles de ser estudiados y caracterizados.

\section{Metodología}
Estudio de corte tranversal retrospectivo. Se utilizó el registro electrónico del SS

Se requieren 3 intervenciones cada turno con distintos fines:

\begin{itemize}
  \item \textbf{Primera intervención:} Presentar el instrumento, resolver dudas y crear correo electrónico y cuentas Google a quienes no las tengan. 2 horas.
  \item \textbf{Segunda intervención:} Acompañar en el primer llenado de datos de los funcionarios. 3 horas.
  \item \textbf{Tercera intervención:}  Resolver dudas y optimizar el uso de la planilla. 2 horas.
\end{itemize}

Total de horas extras requeridas: 28 horas

Se debe considerar tiempo por parte un profesional y un TP que valide los datos ingresados a final de mes, para asegurar que la información registrada es correcta y pueda ser extrapolable. 6 horas cada mes.

Se requiere que jefatura médica y de enfermería del centro regulador informe a todos los reguladores de esta nueva modalidad de trabajo, haciendo hincapié en los nuevos aspectos propuestos: priorización de la clave, lugar de la clave y problema y sub-problema de la clave. Mientras el protocolo de priorización de las solicitudes busca resolución en el Servicio de Salud, se debe dar instrucción directa.

\section{Plazos y resultados esperados}
La planilla debe entrar en funcionamiento el 2 de Octubre 2017 en el turno día.
A final de mes se entregará un resumen con resultados de lo obtenido en la planilla.

%\clearpage

\noindent\begin{minipage}{\linewidth}

\section{Anexo 1: Probelmas y subproblemas}
\begin{table}[H]
\centering
\begin{tabular}{ll}
\textbf{1 Accidentes y traumas}                            & \textbf{2 Colapso}                                \\
Accidente múltiples víctimas (cualquier razón)             & Colapso respiratorio o circulatorio. PCR. Asfixia \\
Accidente vehicular o transporte. Atropellos               & OVACE                                             \\
Agresiones, ataques, asalto                                & Convulsiones                                      \\
Ataques animales, mordeduras y picaduras. Lesion ponsoñoza & Inconciente, desmayo. Respirando                  \\
Autoagresión e intento suicidio                            &                                                   \\
Caídas leves                                               & \textbf{3 Conciencia}                             \\
Caídas graves                                              & Comportamientos extraños                          \\
Hemorragias, laceraciones, heridas o trauma profundo       & Compromiso de conciencia                          \\
Politraumatizado o empalado                                & Enfermedad Psiquiátrica o agitación psicomotora   \\
Lesion arma blanca                                         & Lipotimia recuperada                              \\
Lesion arma fuego                                          & Focalidad Neurológica. Compromiso neurologico     \\
Quemaduras o explosiones                                   &                                                   \\
Trauma Cabeza y cuello o TEC                               & \textbf{5 Intoxicacion}                           \\
Trauma abdominal                                           & Estado embriaguez                                 \\
Trauma toraccico                                           & Exposición o fuga materiales, gases peligrosos    \\
Trauma de extremidad (es)                                  & Sobredosis, envenenamientos e intoxicaciones      \\
                                                           &                                                   \\
\textbf{4 Dolor o problemas especificos}                   & \textbf{6 Sin patologia urgencia}                 \\
Abscesos e infecciones locales                             & Alteración en CSV                                 \\
Posible Sepsis                                             & Heridas simples                                   \\
Deshidratacion                                             & Malfunción de sondas o vias                       \\
Enfermedad terminal, dolor de cancer, agonico.             & Malestar general adulto mayor                     \\
Diarrea y vómito                                           & Malestar general adulto                           \\
Dificultad respiratoria leve - moderada (disnea)           & Malestar general lactante                         \\
Dificultad respiratoria grave                              & Malestar general niño                             \\
Dolor de pecho                                             & Mareos o náuseas                                  \\
Dolor o problemas abdominal                                & Niño irritable                                    \\
Dolor o problemas cabeza (cefalea)                         & Paciente preocupado                               \\
Dolor o problemas de cuello, dorso y espalda               & Reacción Vivencial Anormal (RVA)                  \\
Dolor o problemas de la cara                               &                                                   \\
Dolor o problemas de las piernas y brazos                  & \textbf{7 Otros}                                  \\
Dolor o problemas de orejas y ojos                         & Caso social                                       \\
Dolor o problemas dentales                                 & Fallecido                                         \\
Dolor o problemas garganta                                 & Mantención o reposición                           \\
Dolor o problemas génito-urinarios                         & Catastofre (terremoto, incendio, estampida)       \\
Embarazos, partos, abortos. Problemas puérperas            & Servicio, salida preventiva, evento público       \\
Problemas de Diabetes                                      & Solicitud de consejo de salud                     \\
                                                           & Violencia Intra Familiar                          \\
\textbf{8 Traslado secundario}                             & Otros                                             \\
Tr. Exámen o procedimiento                                 &                                                   \\
Tr. Especialista                                           &                                                   \\
Tr. Domicilio                                              &                                                   \\
Tr. Necesidad UPC                                          &                                                   \\
Tr. Rescate                                                &                                                   \\
Tr. Cupo                                                   &                                                   \\
Tr Previsión                                               &                                                   \\
Tr Alta                                                    &                                                   \\
Otro                                                       &                                                  
\end{tabular}
\caption{Problemas y subproblemas}
\label{problemas y subproblemas}
\end{table}


\label{Anexo1}
\end{minipage} 
\clearpage


\section{Anexo 2: Instructivo llenado formulario}

La planilla es online, por lo que el operador debe entender que existe más de una persona trabajando en el mismo documento. Se sugiere al momento de ingresar una salida colocar inmediatamente el “número de salida diaria” (ver más adelante a que corresponde) correspondiente para que otros usuarios que estén usando el documento sepan que esa fila se utilizará.

Se considera por “entrada” todos los datos llenados para una salida de ambulancia.
Como sugerencia, utilizar las teclas “control” + “shift” + V para pegar información contenida en el portapapeles, de esa forma se respetarán los formatos establecido y utilizados en la planilla

Las acciones sobre las planillas serán registradas bajo el usuario registrado de momento, por lo que todo cambio será registrado en el historial

\subsection{Salida diaria: }
Corresponde al número de salida del día. Es un correlativo que debe sumarse consecutivamente por cada turno de personal.
\subsection{Móvil Samu:}
Identificador del número de ambulancia en cuestión
\subsection{Fecha}
Fecha del día. El formato es en dd-mm-aaaa, donde dd es el día en números del 01 al 31; mm es el mes en números del 01 al 12; y aaaa es el año desde el 2017 en adelante
\subsection{Turno}
Es el turno en el cuál se efectuó la salida de ambulancia, no la solicitud de ella.
\subsection{Rop}
Identificación del funcionario responsable del llenado de la entrada en cuestión
\subsection{Base salida}
Es la base de donde está asignado el móvil en cuestión.
\subsection{Nombre, Edad, Rut, Sexo, Telefono}
Datos demográficos del paciente que requiere la atención SAMU
\subsection{Destino y Lugar}
Son los lugares a donde se despachó la ambulancia y el destino final del paciente atendido.
\subsection{Desenlace}
Corresponde que es lo que sucedió con el paciente que requería atención SAMU
\begin{itemize}
\item \textbf{Atencion en el lugar:} Se realizó una atención SAMU y la ambulancia se retiró. Como ejemplo están las constataciones, atención en domicilio que permite que sintomatología remita y no se requiera traslado en ese momento.
\item \textbf{No encuentra:} Personal de ambulancia a pesar de dirigirse al sitio despachado no hace contacto con solicitante
\item \textbf{Rechaza Atención:} Paciente que tiene indicación de traslado pero lo rechaza.
\item \textbf{Traslado Primario:} Traslado realizado por parte de ambulancia, a pesar de maniobras realizadas en el lugar de la clave
\item \textbf{Traslado Secundario:} Traslado que ocurre entre dos centros de salud
\item \textbf{Salida cancelada:} Si bien la ambulancia sale por una clave, la central SAMU define que no se lleve a cabo por alguna razón
\item \textbf{Tras por otro medio:} Al momento de llegar la ambulancia al sitio despachado, el solicitante ya se trasladó por otro medio.
\item \textbf{Otro:} Alguna circunstancia no cubierta por los ítems anteriores
\end{itemize}

\subsection{X8, R	D, Z-17, Z-8, Z-5, Z-1, Z-2, Km inicio, Km final, Conductor, TP, M23 y Medico}
Corresponden a datos propios de la solicitud y salida de ambulancia. Incluye REM (identificador SAMU), horarios de llegada y salida de ambulancia, kilometrajes de inicio y final de carrera y tripulación que lleva ambulancia

\subsection{Tipo Móvil}
Es la tripulación que aborda la ambulancia. Toda ambulancia tiene conductor y técnico paramédico (TP), al momento de adicionar más personal cambia de su categoría Básica
\begin{itemize}
\item \textbf{Avanzada:} Al TP se suma enfermero
\item \textbf{Medicalizada:} Al TP se suma médico
\item \textbf{X5 o Completa:} Al TP se suma enfermero y médico
\end{itemize}


\subsection{Problema llamada y Sub problema}
Son diferenciaciones categorizadas en niveles y subniveles que describen el motivo de llamada. Se incluye un anexo con todos los problemas y subproblemas

\subsection{Lugar}
Es el lugar donde ocurre la solicitud:
\begin{itemize}
\item \textbf{Vía pública:} Conocido como la calle, acera, vereda. 
\item \textbf{Recinto Público:} Es un lugar abierto concurrido por varias personas, puede ser público o privado. Pueden se bancos, galerías, Municipalidad, colegios.
\item \textbf{Domicilio:} Casa particular que funcione como domicilio
\item \textbf{Centro salud:} Un centro de salud que posea Urgencia o pacientes hospitalizados. Corresponde a SAPU, Hospital, Clínica. No corresponde a asilos, casas acogida, consultorios médicos.

\end{itemize}


\subsection{Prioridad}
Es la priorización otorgada por el regulador médico o de enfermería. Va desde S1 a S5

\begin{table}[H]
\centering
\label{my-label}
\begin{tabular}{c l p{9cm} }
\textbf{Priorizacion} & \textbf{Tiempo llegada} & \textbf{Descripci\'on} \\
\rowcolor[HTML]{EFEFEF} 
$S{_1}$                    & Inmediato                       
& Requiere de reanimaci\'on avanzada inmediata. Cursa con situación potencial o real de riesgo vital o que tiene riesgo de secuelas funcionales graves. Necesita intervención antes de iniciar el traslado. Si lo que se está priorizando es un traslado, es por que urge movilizar al paciente al destino propuesto. Despacho y llegada al lugar debe ser lo antes posible sin ningún tipo de retraso.
%Se debe conseguir cualquier tipo de ayuda para revertir situaci\'on. %\textit{Ejemplos: PCR, insuficiencia respiratoria grave, OVACE, accidente vehicular que impresiona pacientes rojos}
\\
$S{_2}$                    & Rapidamente                       
& Requiere equipo que realice una intervención urgente debido a la alta probabilidad de caer en situación de riesgo vital con secuelas permanentes. No se debe retrasar despacho ni llegada al lugar.
%El paciente requiere manejo urgente, su patolog\'ia debe ser atendida en un plazo breve. Es probable que requiera manejo inicial antes de trasladar a centro asistencial si corresponde. No se debe retrasar despacho ni llegada al lugar. 
%\textit{Ejemplos: Crisis asm\'atica grave, ca\'idas de altura, compromiso de conciencia severo que no responde a est\'imulos pero respira, sangrado no controlable, convulsiones activas, hipoglicemias.}                      
\\
\rowcolor[HTML]{EFEFEF} 
$S{_3}$                    & Pronto                        
& Situación de salud con potencialidad de evolucionar a grave si no tiene evaluación sanitaria e intervención en el mediano plazo. Sin requerimiento de reanimaci\'on actual. Su despacho y llegada al lugar pueden ser influidas por la contingencia.
%\textit{Neumonia, crisis asm\'atica o EPOC, lipotimias, hemorragias digestivas, accidentes automovil\'isticos con confirmaci\'on de que no hay pacientes rojos ni amarillos}                       
\\
$S{_4}$                    & En alg\'un momento                        
& No es urgencia, no requiere movil de reanimaci\'on. Solicitud necesita atenci\'on sanitaria dentro de algunas horas. No es prioridad inmediata pero es una solicitud que debe ser atendida. Su despacho estará condicionado a la contingencia del resto de solicitudes.
%\textit{Cat\'eter urinario disfuncional, Trombosis venosa profunda, dolor importante sin causa clara, compromiso cualitativo de conciencia.}                       
\\ 
\rowcolor[HTML]{EFEFEF} 
$S{_5}$                    & Diferible                      
& Solicitud sin gravedad, no es una urgencia o emergencia. La persona puede estar varias horas a la espera sin riesgo de secuelas funcionales. Esta solicitud podr\'ia incluso no responderse y derivar a otro servicio clínico. 
%\textit{Caso social, compromiso del estado general, dolor o problemas espec\'ificos como diarreas, v\'omitos .}                       
\\
%\textit{SD}                & Sin despacho            
%& Solicitud no presenta criterios para env\'io de ambulancia, solicitante debe conseguir por sus medios traslado para recibir asistencia sanitaria en caso de que lo amerite.                        
\end{tabular}
\caption{Categorizaci\'on de solicitudes}
\end{table}

\subsection{Crítico?}
Pendiente por definir

\subsection{Clave salida y Comentario}
Corresponden a la clásica clave SAMU y un comentario por parte del Radiooperador que puede ser opcional.

\section{Resultados a la fecha}

En contexto de la versión de prueba de la planilla, se recolectaron datos algunos días, logrando con ellos conseguir los siguientes datos preeliminares en 2 días.

\begin{quotation}



Dentro de la muestra se encontraron 22 mujeres y 18 hombres. El promedio de edad de los pacientes atendidos fue de 55.58 años y una desviación standard de 28.32403, la máxima de 94 y una mínima de 7 años.

En el periodo estudiado se encontraron 45 despachos de ambulancias. La cantidad de salidas por base fueron 21 (46.66667\%) para Viña del Mar, 18 (40\%) para Quillota, 2 (4.444444\%) para Quintero, 1 (2.222222\%) para La Ligua, 1 (2.222222\%) para SAPUs, 1 (2.222222\%) para ambulancias privadas y 1 (2.222222\%) para Otra SSVQ

Luego del despacho de ambulancia al lugar de la clave, 34 (75.55556\%) pacientes fueron atendidos en el lugar a donde se despacho la ambulancia y no fueron trasladados, 1 (2.222222\%) fueron trasladados hacia la urgencia respectiva, 0 (0\%) rechazaron atención y no fueron trasladados a pesar de tener indicación de traslado, 0 (0\%) salidas no se encontró el solicitante, y en 0 (0\%) salidas el solicitante se había trasladado de alguna forma al llegar la ambulancia al lugar de la solicitud. 

De las ambulancias despachadas, 1 (2.222222\%) correspondían a ambulancias como equipo de reanimación (médico, enfermermero y técnico paramédico), 3 (6.666667\%) ambulancias medicalizadas (Médico y técnico paramédico), 20 (44.44444\%) ambulancias avanzadas(enfermero y técnico paramédico), y 21 (46.66667\%) ambulancias básicas.

La categoría de las salidas asignadas no se logró registrar en 8 (17.77778\%) ocasiones, pero de las que efectivamente se registraron se cuentan con 11 (24.44444\%) solicitudes S1, 6 (13.33333\%) solicitudes S2, 7 (15.55556\%) solicitudes S3, 7 (15.55556\%) solicitudes S4 y 6 (13.33333\%) solicitudes S5.
               
\begin{table}[h]
\centering

\begin{tabular}{llllll}
              & S1 & S2 & S3 & S4 & S5 \\
X5 o completa & 0  & 1  & 0  & 0  & 0  \\
Medicalizado  & 3  & 0  & 0  & 0  & 0  \\
Avanzado      & 8  & 3  & 1  & 4  & 1  \\
Básico        & 0  & 2  & 6  & 3  & 5 
\end{tabular}
\caption{Tripulación versus prioridad}
\label{my-label}
\end{table}

\begin{table}[h]
\centering
\captionsetup{justification=centering,margin=2cm}
\begin{tabular}{llllll}
     & S1 & S2 & S3 & S4 & S5 \\
Prof & 11 & 4  & 1  & 4  & 1  \\
Bas  & 0  & 2  & 6  & 3  & 5 
\end{tabular}
\centering
\caption{Ambulancias profecionalizada versus prioridad. \\
Valor de p para la prueba exacta de Fisher es 0.0002656566}
\label{my-label}
\end{table}

Finalmente se puede hacer un análisis de los tiempos de salida por prioridad asignada.

\begin{table}[]
\centering
\captionsetup{justification=centering,margin=2cm}
\begin{tabular}{l|lll|lll}
   & \multicolumn{3}{l}{Recibido - Despacho} & \multicolumn{3}{l}{Despacho - Salida} \\
   & Min         & Prom         & Max        & Min        & Prom        & Max        \\ \hline
S1 & 0           & 14           & 48         & 0          & 1,8         & 4          \\
S2 & 0           & 18           & 80         & 1          & 1,8         & 3          \\
S3 & 10          & 33           & 76         & 1          & 3,4         & 8          \\
S4 & 5           & 51           & 162        & 2          & 9,5         & 24         \\
S5 & 5           & 60           & 162        & 2          & 6,6         & 17        
\end{tabular}
\caption{Diferencias de tiempo en minutos entre la recepción de la llamda, \\
el despacho de ambulancia y su salida de la base}
\label{my-label}
\end{table}


    \end{quotation}
    
\end{document}

